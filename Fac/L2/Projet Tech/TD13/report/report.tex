\documentclass{article}

\usepackage[T1]{fontenc}
\usepackage[utf8]{inputenc}
\usepackage[french]{babel}
\usepackage[table]{xcolor}
\usepackage{graphicx, times, listings, hyperref, biblatex, multirow}
\addbibresource{biblio.bib}

\title{\Huge World of Warcraft}
\author{Thomas-Alexandre Moreau aka \textit{Tøxik}}
\hypersetup{
    pdftitle={World of Warcraft},
    pdfauthor={Tøxik},
    pdfpagemode=FullScreen,
}

\begin{document}

\maketitle

\begin{center}
    \includegraphics*[height=4.5cm]{WOW logo.png}
\end{center}


\pagebreak

\begin{abstract}
    Je vais vous présenter ici un jeu-vidéo unique qui m'a marqué tout au long de ma vie,
    et qui continue encore de le faire, un jeu dont je vais vous raconter mon histoire :
    \textbf{World of Warcraft}.
\end{abstract}

\pagebreak

\tableofcontents

\pagebreak

\section{Un MMORPG ? C'est quoi ?}

\subsection{Signification et concept}

<< MMORPG >> signifie << Massively Multiplayer Online Role-Play Game >>, c'est à dire un jeu qui à
pour but de nous faire voyager dans un monde avec une infinité d'autres joueurs en ligne, et
de vaincre les dangers avec eux.\newline
De plus, le but d'un MMORPG est de nous faire incarner notre propre personnage. Les
possibilités d'identification à notre personnage dépendent de la personnalisation
possible de notre personnage dans les différents jeux.

\subsection{Exemples}

Il existe un nombre incalculable de MMORPGs, mais seulement un nombre très réduit ont réussi
à conserver leur popularité. En voici quelques uns : \bigskip

\begin{itemize}
    \item \href{https://fr.finalfantasyxiv.com/}{\textbf{FINAL FANTASY XIV}}\medskip

          \includegraphics*[height=6cm]{FFXIV logo.png}

          Le concurrent direct de World of Warcraft. La licence
          Final Fantasy est une licence très populaire éditée par Square Enix, un studio
          japonais connu dans le monde entier pour ses jeux très appréciés. \pagebreak

    \item \href{https://www.playlostark.com/fr-fr/}{\textbf{LOST ARK}}\medskip

          \includegraphics*[height=6cm]{LA logo.jpg}

          Un jeu beaucoup plus récent qui a énormément fait parler de lui à sa sortie.
          Initialement sorti en Asie il y a plusieurs années, il a fait son apparition
          assez recemment en Occident. Il se démarque par son style assez hardcore et
          élitiste demandant une maitrise parfaite de sa classe à un certain niveau.\medskip

    \item \href{https://www.guildwars2.com/fr/}{\textbf{GUILD WARS 2}}\medskip

          \includegraphics*[height=7cm]{GW2 logo.png}

          Guild Wars 2, successeur du très prometteur Guild Wars 1 à son époque, s'est
          démarqué par son style de jeu assez unique pour un MMORPG, ainsi qu'à son mode
          PvP (Player vs Player ou Joueur contre Joueur (JcJ)) très développé. Bien
          maîtrisé, le jeu prend un tout autre sens en terme de mécaniques, comme pour
          Lost Ark.
\end{itemize}

\pagebreak

\section{World of Warcraft, un MMORPG unique}

Passons à présent au sujet principal, World of Warcraft (ou WoW), le MMORPG qui les
    {\large \textbf{domine}} tous...

\subsection{Un studio mondialement connu}

World of Warcraft est édité par Blizzard Activision, deux très grosses entreprises ayant
fusionnées. Je parlerai ici uniquement de Blizzard Entertainment, car c'est
l'entreprise à l'origine de World of Warcraft.

Initialement \textbf{Silicon \& Synapses, Inc}, puis ensuite \textbf{Chaos Studios, Inc}
et enfin \textbf{Blizzard Entertainment}, ce studio a marqué de multiples générations grâce
à ses licences toutes plus célèbres les unes que les autres : Warcraft, StarCraft, Diablo...

Mais si Blizzard a eu autant de succès, c'est notamment grâce à un jeu, sur la base de
l'univers des jeux Warcraft : \textbf{World of Warcraft}.

\subsection{L'univers}

L'univers de World of Warcraft, développé depuis bien avant sa sortie en 2005, reprend
certaines bases classiques des RPGs (souvent inspirés de Dungeons and Dragons),
des factions en guerre, différentes races telles que les humains et les orcs, etc.

Cependant celui-ci s'est démarqué de tous les autres, c'est un univers extrêmement vaste et
développé par plusieurs sources différentes, car en effet, Warcraft n'est pas à l'origine de
toute l'histoire.

Comme beaucoup d'oeuvres telles que des jeux ou des films, World of Warcraft prend la
majeure partie de son histoire dans les livres, certains étant des romans assez classiques
très bien écrits (généralement par la romancière
\href{https://fr.wikipedia.org/wiki/Christie_Golden}{Christie Golden}), des bandes-dessinées
ou bien de colossales oeuvres dont la source même de l'histoire de Warcraft y est écrite :
les Chronicles. \cite{chronicles-v1}\cite{chronicles-v2}\cite{chronicles-v3}

\subsection{Les classes et races jouables}

Comme tout MMORPG, sur World of Warcraft on a la possibilité de créer son propre
personnage, et ainsi lui donner l'identité que l'on souhaite.\newline
Chaque race a accès à un certain nombre de classes jouables, par exemple : un Mort-vivant
peut jouer Guerrier, mais pas Paladin.\newline\newline
Voici un tableau des classes jouables par chaque race :

\begin{tabular}{|m{2cm}||m{1cm}|m{1cm}|m{1cm}|m{1cm}|m{1cm}|m{1cm}|}
    \hline
    \multicolumn{7}{|c|}{Alliance}                                                    \\
    \hline
    \hline
    Races / Classes      & Humain & Nain & Elfe de la nuit & Gnome & Draeneï & Worgen \\
    \hline
    \hline
    Guerrier             & X      & X    & X               & X     & X       & X      \\
    \hline
    Paladin              & X      & X    &                 &       & X       &        \\
    \hline
    Chasseur             & X      & X    & X               & X     & X       & X      \\
    \hline
    Voleur               & X      & X    & X               & X     & X       & X      \\
    \hline
    Prêtre               & X      & X    & X               & X     & X       & X      \\
    \hline
    Chaman               &        & X    &                 &       & X       &        \\
    \hline
    Mage                 & X      & X    & X               & X     & X       & X      \\
    \hline
    Démoniste            & X      & X    &                 & X     &         & X      \\
    \hline
    Moine                & X      & X    & X               & X     & X       & X      \\
    \hline
    Druide               &        &      & X               &       &         & X      \\
    \hline
    Chasseur de démons   &        &      & X               &       &         &        \\
    \hline
    Chevalier de la mort & X      & X    & X               & X     & X       & X      \\
    \hline
\end{tabular}\bigskip

\begin{tabular}{|m{2cm}||m{1cm}|m{1cm}|m{1cm}|m{1cm}|m{1cm}|m{1cm}|}
    \hline
    \multicolumn{7}{|c|}{Horde}                                                        \\
    \hline
    \hline
    Races / Classes      & Orc & Mort-vivant & Tauren & Troll & Elfe de sang & Gobelin \\
    \hline
    \hline
    Guerrier             & X   & X           & X      & X     & X            & X       \\
    \hline
    Paladin              &     &             & X      &       & X            &         \\
    \hline
    Chasseur             & X   & X           & X      & X     & X            & X       \\
    \hline
    Voleur               & X   & X           & X      & X     & X            & X       \\
    \hline
    Prêtre               & X   & X           & X      & X     & X            & X       \\
    \hline
    Chaman               & X   &             & X      & X     &              & X       \\
    \hline
    Mage                 & X   & X           & X      & X     & X            & X       \\
    \hline
    Démoniste            & X   & X           &        & X     & X            & X       \\
    \hline
    Moine                & X   & X           & X      & X     & X            & X       \\
    \hline
    Druide               &     &             & X      & X     &              &         \\
    \hline
    Chasseur de démons   &     &             &        &       & X            &         \\
    \hline
    Chevalier de la mort & X   & X           & X      & X     & X            & X       \\
    \hline
\end{tabular}\bigskip

Il existe également deux races neutres, c'est à dire qui sont accessibles dans les deux
factions du jeu (l'Alliance et la Horde) : les Pandarens et les Dracthyrs.\newline
Les dracthyrs ont une classe unique disponible pour eux uniquement : l'Evocateur.

\subsection{Les extensions}

Il y a eu un bon nombre d'extensions depuis le début du jeu. On en compte aujourd'hui un
total de 9 :

\begin{description}
    \item[Burning Crusade] La première extension du jeu, celle qui nous a fait découvrir
        un territoire totalement nouveau, en dehors de notre planète d'origine Azeroth. Nous
        Devions vaincre des armées de démons en Outreterre et mettre fin aux agissements
        d'Illidan, le premier chasseur de démons.
    \item[Wrath of the Lich King] Ici retour en Azeroth, mais sur un nouveau continent :
        le Norfendre. Cette zone glacière infestée de morts-vivants et d'aberrations de tout
        genre menaçait de détruire nos royaumes, et ce à cause de l'impitoyable Roi-Liche, un
        ancien paladin ayant succombé aux murmures de son épée Deuillegivre.
    \item[Cataclysm] Les royaumes menacent de s'effondrer à cause d'une nouvelle
        menace. Un dragon titanesque répand feu et sang région après région. Cependant,
        nous avions des alliés de taille pour l'anéantir : les Aspects draconiques.
    \item[Mists of Pandaria] Dans cette extension à l'ambiance paisible se déroulant sur
        un nouveau continent nommée la Pandarie, on y découvre une race bien étrange
        ressemblant à des pandas humanoïdes. Mais en propageant la guerre entre l'Alliance
        et la Horde sur leur territoire, nous avions réveillé un mal ancien, qui semait
        alors le chaos et le désespoir parmis ces populations locales.
    \item[Warlords of Draenor] Nous voici dans un nouveau monde, précisément une réalité
        alternative de l'Outreterre avant les événements menant à l'invasion des démons. On y
        retrouve les peuples féroces étant les ancêtres des orcs que l'ont connait bien.
        Et nous n'y sommes pas le bienvenue.
    \item[Legion] La Legion ardente de démons menée par le Titan corrompu Sargeras fait
        son grand retour sur un nouveau continent, les Iles brisées. On y retrouve
        beaucoup de nouvelles races inconnues qui nous aideront (ou non) à terasser la
        Legion.
    \item[Battle for Azeroth] La guerre entre la Horde et l'Alliance n'a jamais été
        aussi présente. Un minerai inconnu ayant surgit après la défaite de Sargeras force
        les deux factions à une course à la richesse. Cependant, ce mystérieux minerai
        provoque des dégats colossaux à Azeroth, et celle-ci devient alors vulnérable
        à un mal très ancien trop bien connu, les Dieux très anciens, incarnations mêmes de
        l'Ombre et du Vide.
    \item[Shadowlands] Direction le royaume de la Mort dans cette extension, après qu'une
        brèche ait été créée dans le ciel. On devait alors mettre fin aux agissements
        d'un nouvel ennemi qui tire les ficelles dans tout ce royaume. On y découvre plusieurs
        peuples bien mystérieux tels que les Venthyrs ou les Kyrians.
    \item[Dragonflight] La dernière et actuelle extension. Ici il n'est pas question de
        démons ou de Dieux très anciens. Nous retraçons le passé d'un peuple disparu descendant
        du dragon déchu Aile de Mort, autrefois Neltharion, que nous avions vaincu lors de
        l'extension Cataclysm. Mais des menaces élémentaires viennent compromettre leur
        survie.
\end{description}

\subsection{Contenus}

Parlons maintenant des différents types de contenus présents dans le jeu :

\subsubsection{PvE (Player vs Environment)}

Le mode de jeu le plus classique, présent dans tout MMORPG. Il consiste à vaincre le script
du jeu, c'est à dire affronter de nombreux ennemis à travers des raids ou même des
donjons. Mais pour cela il faut acquérir de l'équipement grâce à l'exploration, ou divers
contenus en monde ouvert permettant de récupérer des pièces d'équipement.\newline
Il existe plusieurs modes de difficulté. Pour les raids, nous avons les difficultés
croissantes suivantes : Recherche de Raid / Normale / Héroique / Mythique. Pour les donjons,
c'est plus compliqué : il existe les difficultés Normale / Héroique / Mythique / Mythique +.
Cette dernière est un peu particulière. Elle reprend la base des donjons Mythiques mais
en rajoutant un chronomètre pour finir le donjon et des affixes ayant pour but de
rajouter des mécaniques dans le donjon. Plus le niveau du donjon est haut, plus il sera
difficile et plus il aura d'affixes. Chaque semaine, la rotation des affixes change.\newline
Tout ceci en fait donc le mode le plus célèbre pour les joueurs de haut niveau.

\subsubsection{PvP (Player vs Player)}

Passons maintenant à la catégorie totalement opposée. Ici on ne cherche pas à s'unir entre
groupes de joueurs pour vaincre des boss de raid, mais à mener une équipe de quelques joueurs
pour vaincre d'autres joueurs. Il y a plusieurs formats disponibles : les arènes 2v2 ou 3v3,
les champs de bataille et le PvP en monde ouvert, sans limite de joueurs.\medskip

Que ce soit pour le PvE ou le PvP, nous avons besoin d'optimiser nos dégats pour vaincre nos
ennemis. Cela passe par l'optimisation de nos statistiques (Hâte, Coup critique, Polyvalence
et Maîtrise) ou bien de nos sorts et notre façon de jouer.\newline
Voici une formule représentant les dégats par seconde (DPS) :\newline
\indent$ ((MinWeaponDamage + MaxWeaponDamage) / 2) / WeapondSpeed $\bigskip

Chaque arme possède des dégats minimums et maximums, donc chaque coup infligé sera d'une
valeur aléatoire comprise entre MinWeaponDamage et MaxWeaponDamage. De plus, nous avons une
vitesse qui dépend du type d'arme. Par exemple, une dague infligera plus de coups par seconde
qu'une hache à deux mains.\medskip
Nous pouvons également transformer cette formule en fonction :
\begin{lstlisting}[language=C, numbers=left, frame=single]
    double damagePerSecond(double minDmg, double maxDmg, 
        double speed) {
        return ((minDmg + maxDmg) / 2) / speed;
    }
\end{lstlisting}

\pagebreak

\printbibliography[heading=bibintoc,title={Bibliographie}]

\end{document}