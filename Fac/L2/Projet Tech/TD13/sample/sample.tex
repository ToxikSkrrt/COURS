\documentclass{article}

\usepackage[T1]{fontenc}
\usepackage[utf8]{inputenc}
\usepackage[french]{babel}
\usepackage{graphicx}
\usepackage{listings}


\title{Mon article}
\author{Mr X.}

\begin{document}

\maketitle

\begin{abstract}
    Ceci est le résumé de l'article.
\end{abstract}

\tableofcontents

\section{Première section}

Les résultats sont présentés dans la section~\ref{resultats}.

\subsection{Sous-section 1}

Ouais ouais ouais \cite{preparata-1988}

\begin{lstlisting}[language=C, numbers=left, frame=single]
    void uwu(char *bebou) {
        if(bebou == "aime") {
            return "gnogni";
        }
        return "pas gnogni";
    }
\end{lstlisting}

\subsubsection{sous-sous section}

\subsubsection{sous-sous section}

\begin{description}
    \item[BE] item 1
    \item[B] item 2
    \item[OU] item 3
\end{description}


\subsection{Sous-section 2}

L’équation de la droite est $3x - 2y + 4 = 0$.

\includegraphics[height=1.5cm]{gnu.png}
\scalebox{-1}[1]{\includegraphics[height=1.5cm]{gnu.png}}

\begin{tabular}{|l|c|r|}
    \hline
    Ceci       & est un     & tableau.   \\
    \hline
    \hline
    alignement & alignement & alignement \\
    \hline
    à          & au         & à          \\
    \hline
    gauche     & centre     & droite     \\
    \hline
\end{tabular}

\section{Deuxième section}


\section{Résultats}
\label{resultats}

\bibliographystyle{plain}
\bibliography{biblio}
\end{document}

